\documentclass[12pt]{jsarticle}
\usepackage{amsmath,amssymb}
\usepackage[dvipdfm]{graphicx}
\pagestyle{empty}
\begin{document}
\setlength{\baselineskip}{22pt}
\begin{flushleft}
{\huge 2}
\begin{description}
\item[(1)]半径1の円に内接する6個の半径の等しい円を図1のように描く,
さらに図2のように6個の小さな半径の等しい円を描く,
この操作を無限にくり返したとき,
6個ずつ次々に描かれる円の面積の総和$S_2$と,
それらの円の円周の長さの総和$C_2$を求めよ.
\item[(2)](1)で6個の円を次々に描いていった.
一般に,自然数$n \geqq 2$に対して$3n$個の円を用いて同様の操作を行うとき,
描かれる円の面積の総和$S_n$と,
それらの円の円周の長さの総和$C_n$を求めよ.
\item[(3)]数列$S_2$,$S_3$,$S_4$,$\cdots$の極限値を求めよ.
\end{description}
\end{flushleft}
\includegraphics[width=5cm]{fig_1999_b_1.jpg}
\includegraphics[width=5cm]{fig_1999_b_2.jpg}
\end{document}
