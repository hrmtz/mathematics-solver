\documentclass[12pt]{jsarticle}
\usepackage{amsmath,amssymb}
\pagestyle{empty}
\begin{document}
\setlength{\baselineskip}{22pt}
\begin{flushleft}
{\huge 3}
 半径10の円$C$がある.
半径3の円板$D$を,
円$C$に内接させながら,
円$C$の円周に沿って滑ることなく転がす.
円板$D$の周上の一点を$P$とする.
点$P$が,
円$C$の円周に接してから再び円$C$の円周に接するまでに描く曲線は,
円$C$を2つの部分に分ける.
それぞれの面積を求めよ.
\end{flushleft}
\end{document}
