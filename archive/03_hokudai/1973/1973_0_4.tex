\documentclass[12pt]{jsarticle}
\usepackage{amsmath,amssymb}
\pagestyle{empty}
\begin{document}
\setlength{\baselineskip}{22pt}
\begin{flushleft}
{\huge 4}
\begin{description}
\item[(1)]点$(k, \, 0)$ $(|k|<3)$を通り,
$x$軸に垂直なだ円$\displaystyle\frac{x^2}{9}+\frac{y^2}{4}=1$の弦がある.
この弦を直径とする円の方程式を求めよ.
\item[(2)]1点$P(a, \, b)$が与えられたとき,
(1)で求めた円のうちで点$P$を通るものが存在するための必要十分な条件を求め,
さらに点$P$の存在する領域を図示せよ.
\end{description}
\end{flushleft}
\end{document}
