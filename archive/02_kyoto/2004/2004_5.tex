\documentclass[12pt]{jsarticle}
\usepackage{amsmath,amssymb}
\pagestyle{empty}
\begin{document}
\setlength{\baselineskip}{22pt}
\begin{flushleft}
{\huge 5}
 複素数$\alpha$に対してその共役複素数を$\bar{\alpha}$であらわす.
$\alpha$を実数ではない複素数とする.
複素平面内の円$C$が1,-1,$\alpha$を通るならば,
$C$は$\displaystyle -\frac{1}{\bar{\alpha}}$も通ることを示せ.
(注意:複素平面のことを複素数平面ともいう)
\end{flushleft}
\end{document}
