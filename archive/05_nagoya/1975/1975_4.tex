\documentclass[12pt]{jsarticle}
\usepackage{amsmath,amssymb}
\pagestyle{empty}
\begin{document}
\setlength{\baselineskip}{22pt}
\begin{flushleft}
{\huge 4}
 曲線$y=x^2(x+1)$と直線$y=k^2(x+1)$ $(0 \leqq k \leqq 1)$とで囲まれる部分の面積が最小となるように$k$の値を定めよ.
\end{flushleft}
\end{document}
